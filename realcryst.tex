\section{Изучение однородности монокристаллов твердого раствора оксида иттрия-европия}

При выборе рефлекса, подходящего для уточнения ПЭЯ \YEu{}, мы столкнулись с проблемой оценки его интенсивности.
Так, например, теоретические значения $F$ рефлексов \hkl(6 8 20) и \hkl(8 6 20), с одинаковым угловым положением $2\theta$, соотносятся как $\approx 7/1$.
Естественно, предпочтительно использовать наиболее интенсивное отражение.
Для решения этой проблемы предварительная съемка кристалла была скорректирована --- расстояние $D$ уменьшено до $60\unit{мм}$, а углы $2\theta_D$ увеличены до $\pm 75\degree$.
В результате были построены сечения обратного пространства (см. рис.~\ref{fig:precession}), захватывающие область углов $2\theta \range{95}{100}\degree$.
Сопоставление интенсивностей рефлексов с результатами вычислений программы James позволило выбрать оптимальные индексы.
По такой схеме было проведено исследование 5 монокристаллов, результаты представлены в табл.~\ref{tab:YEu}. Значения ПЭЯ лежат в интервале от $10.6902\unit{\AA}$ до $10.7045\unit{\AA}$, разница крайних значений составляет $0.0143\unit{\AA}$, что значительно превосходит абсолютную погрешность определения ПЭЯ равную $0.0007\unit{\AA}$.
Таким образом, можно однозначно утверждать, что синтезированный продукт не однороден.

\begin{figure}[ht!]
    \centering
    \includegraphics[width=0.8\textwidth]{precession.png}
    \caption{
        Сечение обратного пространства \hkl(6 k l) для кристалла №5.
        Анализ интенсивности рефлексов показывает, что следует использовать рефлекс \hkl(6 8 20).
        Показана дуга, проведенная для $d = 0.5\unit{\AA}$
    }%
    \label{fig:precession}
\end{figure}

Для оценки соотношения Y/Eu в изученных монокристаллах можно использовать правило Вегарда.
Для построения соответствующей прямой были использованы литературные данные~\cite{Swanson:1954,Morris:1984,Nikolaev:2023}, а также результаты проведенного нами РСтА 5 кристаллов, отобранных из того же самого продукта синтеза, из которого отобран кристалл $C$ (см. табл.~\ref{tab:YEu} Приложения).
Вместо утерянного кристалла №2 был изучен №6.
Расчет стратегии съемки для накопления полного массива данных производился для каждого кристалла автоматически с учетом его симметрии \hkl(m -3) по предварительно определенной матрице ориентации с использованием пакета программ APEX3.
Далее проводили интегрирование экспериментальных интенсивностей и вводили поправки на поглощение.
Структуры решены с помощью программы ShelxT~\cite{Sheldrick:2015:shelxt} и уточнены с ShelxL~\cite{Sheldrick:2015:shelxl} в графическом интерфейсе Olex2~\cite{Dolomanov:2009}.
Параметры атомных смещений были уточнены в анизотропном приближении.
В результате установлено, что все изученные кристаллы изоструктурны и представляют собой твердые растворы \YEu{}, причем смешанными оказываются обе позиции металла.
Для примера приведем результат исследования кристалла №5:
\[ \text{Пр. группа} : I a \bar{3}, \ Z = 16 \]
\[ x = 0.277(10), \ a = 10.69180(10)\unit{\AA}, \ V = 1222.23(3)\unit{\AA}^3 \]
\[ \rho_\text{выч} = 5.669\unit{г/см}^3, \ \mu_\text{выч} = 38.366\unit{мм}^{-1}, \ F(000) = 1845.0 \]
\[ \text{Диапазон сбора данных} : 2\theta = \range{7.624}{62.84}\degree \]
\[ \text{Измерено 4650 отражений} : (-11 \leq h \leq 15, -10 \leq k \leq 15, -15 \leq l \leq 10) \]
\[ \text{Независимых рефлексов} : N_\text{рефл}/N_\text{рефл} [I > 2\sigma (I)] = 343/340 \] 
\[ N_\text{параметров} = 19, \ \text{$S$-фактор по $F^2$} = 1.076 \] 
\[ I > 2\sigma (I) : R_1 = 0.0100, \ wR_2 = 0.0224 \]
\[ \text{Все данные} : R_1 = 0.0101, \ wR_2 = 0.0225 \]
Для других кристаллов результаты уточнения аналогичны:
\[ I > 2\sigma (I) : R_1 = \range{0.0100}{0.0142}, \ wR_2 = \range{0.0224}{0.0313} \]
\[ \text{Все данные} : R_1 = \range{0.0101}{0.0152}, \ wR_2 = \range{0.0225}{0.0316} \]
Полученные ПЭЯ и $x$ приведены в табл.~\ref{tab:YEu} Приложения.

В результате обработки данных построена прямая
\[ x = -39.96 + 3.77 a_\text{эксп} \]
Значение $x$ для кристаллов, изученных нами по методике Бонда, приведены в табл.~\ref{tab:YEu} Приложения, вместе с данными для кристалла C~\cite{Nikolaev:2023}: все значения $x$ укладываются в достаточно широкий интервал от $\range{0.27}{0.40}$.

% \begin{figure}[ht!]
%     \centering
%     \includegraphics[width=\textwidth]{YEu.pdf}
%     \caption{YEu.}%
%     \label{fig:YEu}
% \end{figure}
