\section{Обзор методов}

Перед разработкой самой методики были изучены обзорные статьи~\cite{Lider:2020,Galdecka:2006} с целью поиска базового метода, на котором можно будет построить желаемую методику.
В указанных обзорах приводятся различные рентгеновские дифракционные методы измерения и уточнения ПЭЯ.
Их все можно разбить на три группы:
\begin{itemize}
    \item Полихроматические методы (метод Лауэ)
    \item Методы с монохроматичным, но широко расходящимся пучком (метод косселевских проекций)
    \item Методы с монохроматичным и коллимированным пучком.
\end{itemize}
Среди них выбирался тот, который можно адаптировать под стандартный лабораторный монокристальный дифрактометр.
Такой дифрактометр предполагается оснащенным:
\begin{itemize}
    \item Рентгеновской трубкой с хорошо коллимированным пучком
    \item Как минимум однокружным гониометром для образца
    \item Матричным детектором с регулируемым углом поворота вокруг образца
\end{itemize}
Таким образом, в рассмотрении остаются только методы, использующие монохроматическое и коллимированное излучение, так как используемое в лабораторных дифрактометрах характеристическое излучение можно считать монохроматическим, относительно фонового тормозного.

\subsection{Метод Бонда}

В первую очередь из этих методов выделяется метод Бонда.
Он является простым, безэталонным, универсальным, и позволяющим получить хорошую точность, вплоть до $10^{-6}$.

В оригинальном исполнении установка Бонда~\cite{Bond:1960} представляет собой однокристальный спектрометр.
В качестве источника используется микрофокусная рентгеновская трубка с коллиматором в виде пары пластин дающих расходимость первичного пучка около $0.8'$.
Кристалл --- это ориентированная полированная пластина из монокристаллического кремния, значительно превосходящая размерами первичный пучок.
Для поддержания постоянной температуры в $24.7\celcius$ образца при этом используется водяное охлаждение.
Кристалл был закреплен на гониометре, позволяющим регулировать угол наклона кристалла и вращать его в одной плоскости $\omega$ с точностью до $1''$.
В качестве детекторов использовались два счетчика гейгера, которые считаются точечными детекторами.
Они могли вращаться в той же плоскости, что и кристалл.

Само измерение ПЭЯ в схеме Бонда выглядит так:
\begin{enumerate}
    \item Выбирается плоскость кристалла, отражение от которой будет измеряться
    \item Отражающая плоскость выставляется перпендикулярно первичному пучку
    \item Детектор устанавливается под углом, чтобы зарегистрировать отражение от плоскости
    \item Измеряется зависимость интенсивности на детекторе от угла поворота $\omega$ кристалла вблизи отражающего положения (кривая качания)
    \item Из полученной зависимости определяется угол $\omega_1$ при котором достигается максимум интенсивности на детекторе
    \item Предыдущие три шага повторяются для симметричного положения детектора и определяется второй угол $\omega_2$
    \item Угол дифракции вычисляется как $2\theta=180\degree-|\omega_1-\omega_2|$
\end{enumerate}
Затем уже значения межплоскостных расстояний вычисляются из уравнения Вульфа-Брэгга~(\ref{eq:bragg}), и, зная индексы Миллера отражающих плоскостей и сингонию кристалла, это позволяет вычислить наконец и значения ПЭЯ.

При определении угла $2\theta$ по такой схеме исключаются ошибки, связанные с:
\begin{itemize}
    \item Смещением образца (эксцентриситетом)
    \item Поглощением в кристалле
    \item Положением нуля гониометра
\end{itemize}
Другие же ошибки имеют незначительное влияние и для них есть выражения, позволяющие вводить поправки для их учета.
Список источников этих ошибок:
\begin{itemize}
    \item Неточное выведение отражающей плоскости параллельно оси вращения $\omega$
    \item Расходимость первичного пучка
    \item Отклонение первичного пучка от плоскости $\omega$
    \item Преломление в кристалле
    \item Фактор Лоренца-поляризации
    \item Ошибка измерения угла
\end{itemize}

В оригинальной работе Бонда ему удалось достигнуть относительной погрешности определения ПЭЯ около нескольких частей на миллион, то есть порядка $10^{-6}$.

\subsection{Модификации метода Бонда}

Схема Бонда была адаптирована и для изучения малых монокристаллов~\cite{Hubbard:1976,Ponomarev:1969}.
Для этого было предложено дополнительно измерять угол отражения от фриделевской пары выбранной плоскости.
Углы $\omega$ для фриделевской пары плоскостей отличаются в таком случае на $180\degree$, и это позволяет уменьшить ошибку калибровки гониометра.
Остальные ошибки уменьшаются за счет учета ассиметричности кривой качания.
Она смещает измеряемое значение угла $\omega$, но при использовании фриделевской пары, это смещение в оказывается преимущественно направлено в разные стороны.
Таким образом, полусумма углов $\omega$ для фриделевской пары плоскостей позволяет в среднем уменьшить погрешность определения ПЭЯ.

Для трехкружного гониометра возможно использование методики измерения от одной плоскости уже для 8 различных углов гониометра~\cite{King:1979}.
В такой схеме можно учесть еще больше ошибок, связанных со смещением образца от точки сведения осей гониометра, а также она позволяет определить нулевые положения гониометра.
Для реализации этого метода даже была написана специальная программа, которые позволяют калибровать дифрактометры с трехкружными гониометрами и точечными детекторами~\cite{Angel:2011}.

\subsection{Метод щелей Соллера}

В работе~\cite{Berger:1984} предлагается метод, использующий всего одно отражающее положение кристалла, но при этом получаемая точность оказывается не хуже чем в методе Бонда, то есть порядка $10^{-6}$.
Угол дифракции же в нем измеряется с помощью щелей Соллера.

В этом методе сначала кристалл и детектор выводятся в отражающее положение, при котором наблюдается максимальная интенсивность дифрагированного луча.
После этого на гониометр устанавливаются щели Соллера, и измеряется угол между их положениями на гониометре, в которых наблюдается максимальная интенсивность.

Из-за использования щелей Соллера, этот метод оказывается невосприимчивым к центрировке образца, и в нем можно использовать источник с большой расходимостью.
По сравнению с методом Бонда он оказывается более сложным, и требует больше специального оборудования для реализации.

\subsection{Метод четырехкристального спектрометра}

Похожий метод измерения угла~\cite{Fewster:1989} использует два кристалла для монохроматизации первичного пучка и один на дифрагированном пучке.
Этот метод, так же как и использующий щели Соллера, невосприимчив к центрировке образца.
Из его недостатков можно отметить, что он требует серьезной модификации установки и тщательной юстировки.

\subsection{Метод компланарных рефлексов}

Этот метод~\cite{Isomae:1976} возможен только для кристаллов, в которых угол между определенными плоскостями однозначно определяется индексами Миллера и не зависит от экспериментально определенных параметров.
При этом измеряется малый угол, на который нужно повернуть кристалл, чтобы вместо отражения от одной плоскости, появилось отражение от второй.
Из-за малого угла поворота, условия съемки практически неизменны для двух отражающих положений, что позволяет свести ошибки, связанные с этим к минимуму.
Этот метод также обладает высокой точностью, на уровне $10^{-6}$.

\subsection{Метод многолучевой дифракции}

Многолучевая дифракция возникает, когда две или более плоскостей оказываются в отражающем положении одновременно.
При этом возможно наблюдение рефлексов, запрещенных кинематической теорией дифракции.
Такое явление впервые наблюдалось Реннингером~\cite{Renninger:1937} и было названо им \textit{Umweganregung} или <<окольным возбуждением>>.
Пики, возникающие в при многолучевой дифракции, получаются более узкими, чем обычные, и поэтому их положение можно измерять более точно.
Но точность такого метода в итоге оказывается не сильно лучше того же метода Бонда: всего лишь на уровне $10^{-5} - 10^{-6}$.

\subsection{Методы эталонов}

Методы эталонов предполагают определение межплоскостных расстояний опираясь не на известную длину волны, а на хорошо известные ПЭЯ эталонного кристалла.
Вообще, эталоны могу использоваться неявно для калибровки дифрактометра, или явно в виде дополнительных, используемых одновременно с образцом кристаллов.
Может быть применен внутренний эталон как в порошковой дифракции, где в исследуемый образец добавляются кристаллы эталонного и снимается дифракционная картина их обоих одновременно.
Использование эталонов само по себе не является конкретной схемой эксперимента, а говорит о внедрении в него дополнительного источника информации.
Явное использование эталонов для измерения ПЭЯ никак не будет использоваться для разработки методики, так как это достаточно сложно и не подходит под поставленные условия.

\subsection{Выбор метода}

Среди всех перечисленных методов в итоге был выбран самый первый --- метод Бонда.
Его простота и универсальность оказываются очень привлекательными и поэтому должны сделать итоговую методику более доступной.