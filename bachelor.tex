\documentclass[a4paper, 12pt]{article}

\usepackage[T2A]{fontenc}
\usepackage[utf8]{inputenc}
\usepackage[english,russian]{babel}
\usepackage[left=3cm, right=1cm, top=2cm, bottom=2cm]{geometry}

\usepackage[acronym]{glossaries}
\newacronym{ucp}{ПЭЯ}{Параметры элементарной ячейки}
\newacronym{powderXRD}{РФА}{Рентгенофазовый анализ}
\newacronym{singleXRD}{РСтА}{Рентгеноструктурный анализ}

\usepackage[hidelinks]{hyperref}

\usepackage{graphicx}
\graphicspath{images}

\author{Кудрявцев А.Л.}
\title{Разработка прецизионного метода определения параметров элементарной ячейки для монокристального дифрактометра, оснащенного двумерным детектором}

\begin{document}
\maketitle
\newpage
\tableofcontents
\newpage
\section{Введение}
\gls{ucp} --- это одна из основных характеристик кристалла. На их значения влияет 
На \gls{ucp} влияет большое число факторов. \\
Поэтому \gls{ucp} --- информация о любом из факторов. \\
Требуется --- высокая прецизионность. \\
Методы рентгеновской дифракции --- измерение \gls{ucp}. \\
Самые распространенные --- \gls{powderXRD} и \gls{singleXRD}. \\
Образец \gls{powderXRD} --- кристаллический порошок. \\
Образец \gls{singleXRD} --- малый монокристалл. \\
Установки рассчитаны --- на один метод. \\
Современный \gls{singleXRD} сравнительно с \gls{powderXRD} --- высокая погрешность \gls{ucp}. \\
Разработанный метод --- низкая погрешность.
\section{Литературный обзор}
\subsection{Обзор методов}
Были изучены обзорные статьти~\cite{Lider:2020,Galdecka:2006}.
В них производятся обзоры рентгеновских дифракционных методов измерения \gls{ucp}.
Среди них выбирался тот, который можно адаптировать под стандартный лабораторный монокристальный дифрактометр.
Такой дифрактометр предполагается оснащенным:
\begin{itemize}
    \item Рентгеновской трубкой с хорошо монохроматизированным и колимированным пучком.
    \item Как минимум моторизированным однокружным гониометром для образца.
    \item Матричным детектором регулируемым углом поворота.
\end{itemize}
Таким образом из всего многообразия методов срзазу исключаются интерференционные, полихроматические, а также использующие сильно расходящийся пучок методы. Также исключаются методы, требующие установки дополнительных монохроматоров и колиматоров.
Среди оставшихся можно выделить методы:
\begin{itemize}
    \item Бонда
    \item Обратного рассеяния
    \item Компланарных рефлексов
    \item Реннингера
    \item Эталонов
\end{itemize}
Метод Бонда среди них --- простой, безэталонный, универсальный в реализации, не имеющий строгих требований и дающий при аккуратном проведении эксперимента очень хорошую точность.
Его идея и взята за основу разработанной нами методики.
\subsection{Метод Бонда}
В оригинальном исполнении~\cite{Bond:1960} схема Бонда представляет собой однокристальный спектрометр.
В качестве источника используется колимированный монохроматизированный пучок.
Кристалл --- это ориентированная монокристаллическая пластинка, размерами превосходящая первичный пучок.
Детектор используется точечный, с возможностью вращаться вокруг той же оси, что и кристалл.
Само измерение угла дифракции в схеме Бонда выглядит так:
\begin{enumerate}
    \item Выбирается плоскость кристалла, отражение от которой будет измеряться
    \item Детектор устанавливается под углом, чтобы зарегистрировать отражение от плоскости
    \item Измеряется зависимость интенсивности на детекторе от угла поворота $\omega$ кристалла вблизи отражающего положения
    \item Из полученной зависимости определяется угол $\omega_1$ при котором достигается максимум интенсивности на детекторе
    \item Предыдущие три шага повторяются для симметричного положения детектора и определяется второй угол $\omega_2$
    \item Угол дифракции вычисляется как $2\theta=180^\circ-|\omega_1-\omega_2|$
\end{enumerate}
Определение угла $2\theta$ по такой схеме является более точным чем по одиночному отражению, так как вычисляя разницу углов $\omega$ исключаются ошибки связанные с эксцентриситетом, поглощением и нулевым положением угла $\omega$.

Схема Бонда была адаптирована и для изучения малых монокристаллов~\cite{Hubbard:1976,Ponomarev:1969}.
В этом случае уже не исключаются ошибки, связанные с эксцентриситетом образца.
Для их конмпенсации изначальную методику дополнили измерением углов $\omega$ отражений для фриделевской пары изначальной плоскости.
Таким образом суммарно для измерения одного межплоскостного расстояния нужно снять профили 4 различных рефлексов.

Для трехкружного гониометра используются методики измерения 8 различных рефлексов~\cite{King:1979}.
В такой схеме можно учесть все ошибки, связанные со смещением образца от точки сведения осей гониометра, а также определить нулевые положения гониометра.

Ключевой особенностью совеременных монокристальных дифрактометров является исплоьзование двумерных детекторов, которое, с одной стороны уменьшает время собора данных для \gls{singleXRD}, а с другой негативно влияет на их качество~\cite{Dudka:2017}.

Методика точного измерения угла дифракции при использовании двумерного детектора по аналогии с оригинальной схемой оказывается во многом не удобной.
В том числе необходимость ручного суммирования сигнала и обработки большого числа снимков.
В качестве альтернативы был выбран метод, использовавшивайся в~\cite{Serebrennikova:2021}.

В этом методе снимается не зависимость интенсивности от угла поворота кристалла $I(\omega)$, а двумерный профиль интенсивности при полном равномерном повороте кристалла вокруг оси $\omega$ через отражающее положение.
В таком случае, вид зависимости интенсивности от координат детектора в основном определяется спектром первичного пучка.
Зная его можно довольно точно определять положения дифракционных пиков на детекторе, из которых в дальнейшем можно определить и углы дифракции.
\section{Экспериментальная часть}
\subsection{Описание установки}
Рентгенографические эксперименты проводились на монокристальном дифрактометре Bruker D8 Venture.
\begin{itemize}
    \item Микрофокусная трубка Incoatec $I \mu S \ 3.0$
    \begin{itemize}
        \item $\text{Cu} K\alpha$ и $\text{Mo} K\alpha$ излучение
        \item Монохроматизация и фокусировка с помощью многослойных зеркал Монтела
        \begin{itemize}
            \item Диаметр пучка $110 \ \text{мкм}$
            \item Расходимость пучка $0.3^\circ$
        \end{itemize}
    \end{itemize}
    \item Двумерный детектором PHOTON III
    \begin{itemize}
        \item Разрешение $768 \times 1024$ пикселей
        \item Размер пикселя $135 \times 135 \ \text{мкм}^2$
        \item Ручная установка расстояния до образца
    \end{itemize}
    \item Трехкружный гониометр FIXED-CHI
    \begin{itemize}
        \item Угол $\chi$ фиксирован и равен $54.7112^\circ$
        \item Паспортная воспроизводимость установки углов $0.0001^\circ$
        \item Паспортная точность установки углов не указана, но согласно результатам измерения эталонного образца на порошковом дифрактометре Bruker D8 Advance, оснащенном аналогичным гониометром, она не хуже $0.005^\circ$
    \end{itemize}
    \item Температурная приставка Oxford Cryostream 800Plus
    \begin{itemize}
            \item Стабильность поддержания температуры $0.2 \ \text{K}$
    \end{itemize}
    \item Управление прибором средствами програмного пакета APEX3~\cite{Bruker:2019}.
\end{itemize}
Необходимо отметить, что из-за расположения трубок область доступных углов для детектора оказывается ограниченой.
Для использовавшегося расстояния от образца до детектора около $130 \ \text{мм}$, угол $2\theta_D$ не мог превосходить примерно $100^\circ$.
\subsection{Описание методики}
Первое описание методики дано в статье~\cite{Kudryavtsev:2024:1}.
Общая схема проведения измерений выглядит примерно так:
\begin{enumerate}
    \item Отбор монокристалла
    \item Предварительная съемка
    \item Выбор рефлекса
    \item Съемка рефлекса
    \item Обработка профилей
    \item Рассчет межплоскостного расстояния
\end{enumerate}
\subsubsection{Отбор монокристалла}
Отбор монокристалла проводится так же, как и для \gls{singleXRD}.
Монокристалл выбирается так, чтобы не превосходить размера первичного пучка.
В нашем случае оптимальный размер равен приблизительно 50~мкм.
\subsubsection{Предварительная съемка}
Предварительная съемка проводится с целью определения ориентации кристалла, его дифракционного класса и получения данных об интенсивности рефлексов.

Сама съемка состоит серии полных сканирований при вращении вокруг оси $\varphi$ с шагом $0.5^\circ$ для при фиксированном угле $\omega$.
Три таких сканирования выполняются при углах детектора $2\theta_D = -45^\circ, 0^\circ, 45^\circ$ при фиксированном расстоянии до образца $D \approx 70 \ \text{мм}$.

Обработка снимков и получение ориентации производится в программме APEX3.
На выходе программы получается файл формата p4p, где информация об ориентации кристалла содержится в виде UB матрицы~\cite{Busing:1967}.
\subsubsection{Выбор рефлекса}
Выбор рефлекса для съемки происходит так, чтобы погрешность измерений была минимальной.
Основными критериями в таком случае оказываются наибольшие угол $2\theta$ и интенсивность рефлекса.
При этом необходимо учитывать геометрию установки, так как не все рефлексы оказывается возможно вывести в отражающее положение для двух симметричных положений в экваториальной плоскости.

Средствами программы APEX3 производить такой перебор рефлексов неэффективно и крайне проблематично, так как программа рассчитывает для одного рефлекса максимум только одну пару углов $(\varphi, \omega)$ из двух возможных в общем случае.
Поэтому была специально написана программа~\cite{Kudryavtsev:2024:2} для перебора всех рефлексов, рассчета для них угов гониометра и отбора случаев когда в оказывается возможным вывести рефлекс в два симметричных положения, а также когда доступна для выведения и его фриделевская пара.

Программа позволяет находить среди множества плоскостей, связанных симметрией такие, которые можно вывести в отражающее положение хотя бы при одном (из двух симметричных) положений детектора.
Для этого используется информация о текущей ориентации кристалла на гониометре, т.е. p4p-файл, в котором находится матрица ориентации UB и предварительные значения ПЭЯ.
Используя известную длину волны, размеры пикселя, расстояние до детектора, и другие неизменные параметры прибора, программа вычисляет углы гониометра $(\varphi, \omega)$, необходимые для выведения каждой плоскости в отражающее положение на экваториальную плоскость.
В каждом случае проверяются геометрические ограничения прибора.
Полученная информация для всех подходящих рефлексов собирается в таблицу Excel, ее можно проанализировать и провести отбор.
\subsubsection{Съемка рефлекса}
Съемка рефлекса представляет собой сканирование при вращении вокруг оси $\omega$ в диапазоне $\pm 2^\circ$ относительно расчитанного значения $\omega$ для отражающего положения.
Время съемки выставлялось таким, чтобы максимум на профиле пика составлял не менее 10000~имп.

В программе APEX3 невозможно выставить время съемки больше 10~мин., поэтому для достижения последнего условия производились несколько одинаковых съемок по 10~мин. пока не достигнется требуемая интенсивность.
\subsubsection{Обработка профилей}
Обработка профилей состоит из нескольких этапов, по завершению которых можно рассчитать межплоскостное расстояние.
Реализована она была тоже в виде программы~\cite{Kudryavtsev:2024:2}.

На входе она использует p4p-файл и информацию о примерном положении центра детектора (результат юстировки, прямое определение, калибровка).
Из экспериментального фрейма вырезается центральная область $X = \pm 30, Y = \pm 15 \ \text{пикс.}$, в которой, исходя из условия $2\theta_D \approx 2\theta$, должен находиться искомый рефлекс.
Медианное значение интенсивности принимается за начальное значение фона.
Пиксели с интенсивностью больше заранее заданной принимаются за "горячие пиксели" и их значения приравниваются среднему значению по 8 соседним пикселям.
После учета горячих пикселей максимум интенсивности в выбранной области назначается примерным положением $K\alpha_1$--составляющей.
Далее, исходя из значений $D$ и $2\theta$ рассчитывается положение $K\alpha_1$--составляющей и обе точки смещаются так, чтобы теоретическое положение $K\alpha_1$ совпадало с координатами найденного максимума интенсивности.
Аппроксимация дублета проводится двумя независимыми  функциями 2D-Gauss, т.е. без закрепления междублетного расстояния и соотношения интенсивностей составляющих $2/1$.
Направлениями главных осей берутся вдоль координат детектора $X$ и $Y$ детектора.
В наших экспериментах именно функция 2D-Gauss наиболее хорошо описывала форму пика при минимальном числе уточняемых параметров: координаты максимума, полуширины (ширина на половине высоты, FWHM) в направлениях $X$ и $Y$, и интегральная интенсивность.
\subsubsection{Рассчет межплоскостного расстояния}

\newpage
\bibliographystyle{unsrt}
\bibliography{bibliography}
\end{document}