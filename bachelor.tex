\documentclass[a4paper, 12pt]{article}

\usepackage[T2A]{fontenc}
\usepackage[utf8]{inputenc}
\usepackage[english,russian]{babel}
\usepackage[left=3cm, right=1cm, top=2cm, bottom=2cm]{geometry}

\usepackage[acronym]{glossaries}
\newacronym{ucp}{ПЭЯ}{Параметры элементарной ячейки}
\newacronym{powderXRD}{РФА}{Рентгенофазовый анализ}
\newacronym{singleXRD}{РСтА}{Рентгеноструктурный анализ}

\usepackage[hidelinks]{hyperref}

\usepackage{graphicx}
\graphicspath{images}

\author{Кудрявцев А.Л.}
\title{Разработка прецизионного метода определения параметров элементарной ячейки для монокристального дифрактометра, оснащенного двумерным детектором}

\begin{document}
\maketitle
\newpage
\tableofcontents
\newpage

\section{Введение}

\gls{ucp} --- это одна из основных характеристик кристалла. На их значения влияет 
На \gls{ucp} влияет большое число факторов. \\
Поэтому \gls{ucp} --- информация о любом из факторов. \\
Требуется --- высокая прецизионность. \\
Методы рентгеновской дифракции --- измерение \gls{ucp}. \\
Самые распространенные --- \gls{powderXRD} и \gls{singleXRD}. \\
Образец \gls{powderXRD} --- кристаллический порошок. \\
Образец \gls{singleXRD} --- малый монокристалл. \\
Установки рассчитаны --- на один метод. \\
Современный \gls{singleXRD} сравнительно с \gls{powderXRD} --- высокая погрешность \gls{ucp}. \\
Разработанный метод --- низкая погрешность.

\section{Обзор литературы}
Были изучены обзорные статьти~\cite{Lider:2020,Galdecka:2006}.
В них производятся обзоры рентгеновских дифракционных методов измерения \gls{ucp}.
Среди них выбирался тот, который можно адаптировать под стандартный лабораторный монокристальный дифрактометр.
Такой дифрактометр предполагается оснащенным:
\begin{itemize}
    \item Рентгеновской трубкой с хорошо монохроматизированным и колимированным пучком.
    \item Как минимум моторизированным однокружным гониометром для образца.
    \item Матричным детектором регулируемым углом поворота.
\end{itemize}

Таким образом из всего многообразия методов срзазу исключаются интерференционные, полихроматические, а также использующие сильно расходящийся пучок методы. Также исключаются методы, требующие установки дополнительных монохроматоров и колиматоров.
Среди оставшихся можно выделить методы:
\begin{itemize}
    \item Бонда
    \item Обратного рассеяния
    \item Компланарных рефлексов
    \item Реннингера
    \item Эталонов
\end{itemize}
Метод Бонда среди них --- простой, безэталонный, универсальный в реализации, не имеющий строгих требований и дающий при аккуратном проведении эксперимента очень хорошую точность.
Его идея и взята за основу разработанной нами методики.

В оригинальном исполнении~\cite{Bond:1960} схема Бонда представляет собой однокристальный спектрометр.
В качестве источника используется колимированный монохроматизированный пучок.
Кристалл --- это ориентированная монокристаллическая пластинка, размерами превосходящая первичный пучок.
Детектор используется точечный, с возможностью вращаться вокруг той же оси, что и кристалл.
Само измерение угла дифракции в схеме Бонда выглядит так:
\begin{enumerate}
    \item Выбирается плоскость кристалла, отражение от которой будет измеряться
    \item Детектор устанавливается под углом, чтобы зарегистрировать отражение от плоскости
    \item Измеряется зависимость интенсивности на детекторе от угла поворота $\omega$ кристалла вблизи отражающего положения
    \item Из полученной зависимости определяется угол $\omega_1$ при котором достигается максимум интенсивности на детекторе
    \item Предыдущие три шага повторяются для симметричного положения детектора и определяется второй угол $\omega_2$
    \item Угол дифракции вычисляется как $2\theta=180^\circ-|\omega_1-\omega_2|$
\end{enumerate}
Определение угла $2\theta$ по такой схеме является более точным чем по одиночному отражению, так как вычисляя разницу углов $\omega$ исключаются ошибки связанные с эксцентриситетом, поглощением и нулевым положением угла $\omega$.

Схема Бонда была адаптирована и для изучения малых монокристаллов~\cite{Hubbard:1976,Ponomarev:1969}.
В этом случае уже не исключаются ошибки, связанные с эксцентриситетом образца.
Для их конмпенсации изначальную методику дополнили измерением углов $\omega$ отражений для фриделевской пары изначальной плоскости.
Таким образом суммарно для измерения одного межплоскостного расстояния нужно снять профили 4 различных рефлексов.
Так как плоскость выводится параллельно оси $\omega$, то для фриделевских пар углы отличаются от соответствующих $\omega_1$ и $\omega_2$ приблизительно на $180^\circ$.

Для трехкружного гониометра используются методики измерения 8 различных рефлексов~\cite{King:1979}.
В такой схеме можно учесть все ошибки, связанные со смещением образца от точки сведения осей гониометра.


\bibliographystyle{plain}
\bibliography{bibliography}

\end{document}