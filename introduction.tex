\section{Введение}

\subsection{Параметры элементарной ячейки}

Параметры элементарной ячейки (ПЭЯ) --- величины, определяющие метрику кристаллической решетки.
Они являются одними из основных характеристик кристаллов.
В общем случае, их представляют в виде шести различных вещественных величин: трех длин, соответствующих периодам главных направлений кристаллической решетки $(a, b, c)$, и трех углов между этими направлениями $(\alpha, \beta, \gamma)$.
Однако, благодаря симметрии кристалла, общее число независимых параметров может быть меньше: минимум одна длина $a$ в случае кубической сингонии.

На ПЭЯ кристалла влияет множество различных факторов, таких как: состав, структура, дефектность, температура, давление и другие.
Это позволяет по изменению ПЭЯ кристалла косвенно их отслеживать.
В общем случае, при значительной разнице в ПЭЯ можно говорить, что кристаллы заметно отличаются, а при достаточной точности измерений и знании того, что может отличаться --- дать количественную оценку изменения этих величин.

Однако, при небольшом их изменении, разница в ПЭЯ крайне мала.
Так, например, температурные коэффициенты расширения большинства материалов порядка $10^{-5} \unit{K}^{-1}$.
Для твердых растворов же, относительная разница значений ПЭЯ для соответствующих чистых веществ порядка $10^{-2}$, и, при изменении мольной доли на $10^{-3}$, относительное изменение ПЭЯ уже составит порядка $10^{-5}$.
Похожим образом ситуация обстоит и с остальными величинами.
Поэтому при использовании ПЭЯ для измерения косвенно связанных с ним характеристик, необходима высокая точность измерений.

Хотя, технически правильнее будет говорить не о точности, а о <<прецизионности>> измерений, от английского \textit{precision}.
В работе эти термины будут отличаться, и разница в них будет заключается в разных ошибках, которым они соответствуют.
Высокая точность будет означать, что полученные значения мало отличаются от истинного значения, а высокая прецизионность --- то, что полученные значения будут мало отличаться друг от друга.
Можно сказать, что для точных результатов систематическая ошибка значительно меньше случайной, а для прецизионности наоборот --- случайная меньше систематической.
В описанном выше случае измерения величин, слабо влияющих на ПЭЯ, систематическая ошибка практически не будет влиять на получаемые результаты, ведь обычно приходится оценивать относительную разницу величины ПЭЯ.

\subsection{Методы измерения}

Наконец, стоит рассказать о самих методах измерения ПЭЯ.
Самыми эффективными и распространенными являются различные дифракционные методы: рентгеновские, нейтронные и электронные.
Среди них самым доступным и неприхотливым является именно рентгеновская дифракция, и только она будет рассматриваться в дальнейшем.
В любой дифракции, тем не менее, основным уравнением, позволяющим, зная длину волны $\lambda$ излучения и угол дифракции $2\theta$, получить межплоскостные расстояния в кристалле $d$ является уравнение Вульфа-Брэгга~(\ref{eq:bragg}).
\begin{equation} \label{eq:bragg} 
    2 d \sin{\theta} = \lambda
\end{equation}

Установками, на которых она реализуется являются обычно лабораторные дифрактометры и специализированные станции синхротронного излучения (СИ).
Между ними, конечно, есть принципиальная разница в источнике излучения, но общая схема установки у них одинаковая.
Они состоят из:
\begin{itemize}
    \item Источника излучения
    \item Исследуемого образца
    \item Детектора излучения
\end{itemize}
В добавок к этому, каждый из этих компонентов может быть механизирован, то есть для них может регулироваться положение и ориентация в пространстве.
В качестве примера, лабораторные монокристальные дифрактометры обычно оснащаются гониометром, с помощью которого возможно точное вращение кристалла.
Так же может быть и механизирован и детектор: может регулироваться расстояние между ним и образцом, а также сам детектор обычно может вращаться вокруг образца.
Перемещение же источника возможно только для рентгеновских трубок, и это обычно реализуется в порошковых дифрактометрах.

Рентгеновские дифракционные методы отличаются между собой, и один из способов их классификации --- по виду образца: монокристальные, поликристальные, порошковые, тонкопленочные и другие.
Стандартным способом точного и воспроизводимого измерения ПЭЯ сейчас является порошковая дифракция, тогда как монокристальная используется в основном для проведения рентгеноструктурного анализа.
Может показаться странным, но данные о ПЭЯ, получаемые сейчас из рентгеноструктурного анализа по монокристальной дифракции часто являются менее достоверными и воспроизводимыми, чем порошковые данные~\cite{Dudka:2017}.
Это связано, в основном, с использованием двумерных детекторов вместо точечных.

\subsection{Двумерные детекторы}

Двумерные или, по-другому, матричные детекторы используют полупроводники для регистрации рентгеновских квантов.
Есть три основных типа таких детекторов: CCD, CMOS и HPAD~\cite{Alle:2016}.

CCD, или \textit{charge-coupled device} --- это прямые аналоги матриц, используемых для съемки в видимом диапазоне.
Рентгеновские кванты попадая на сцинтиллятор детектора преобразуются в фотоны, которые затем преобразуются фотодиодом в электроны и собираются в потенциальные ямы, называемые пикселями детектора.
Затем этот заряд автоматически измеряется электроникой и получается двумерное изображение, на котором интенсивность каждого пикселя напрямую связана с зарядом, накопленном в яме.

CMOS, или \textit{complementary metal-oxide-semiconductor} --- это по-сути усовершенствованная версия CCD-матрицы.
Главное отличие их в том, что в CMOS рядом с каждым пикселем детектора располагается небольшая электронная схема, обрабатывающая получаемый сигнал, усиливая его и нормализуя.
Это делает данные более достоверными, уменьшая ошибки, но увеличивает размеры пикселя, а также уменьшает активную площадь потенциальной ямы для электроном, что в свою очередь уменьшая общую чувствительность детектора.

HPAD, или \textit{hybrid pixel array detector} принципиально отличается от предыдущих двух.
Они основываются на технологии регистрации высокоэнергетических фотонов в физике высоких энергий.
По-сути это матрица из полупроводниковых счетчиков рентгеновских фотонов.
В каждом таком счетчике каждый фотон напрямую в полупроводнике преобразуется в электрон-дырочные пары без промежуточного преобразования в фотоны видимого диапазона.
Такие детекторы точнее, быстрее и эффективнее тех, что использую сцинтилляторы.

\subsection{Мотивация разработки методики}

Теперь учитывая то, что технологии детектирования рентгеновского излучения стремительно развились по сравнению с точечными детекторами, а точность определения ПЭЯ оставляет желать лучшего, то значит, что проблема монокристальной дифракции может заключаться в плохой технике эксперимента или обработке данных.
Матричные детекторы позволяют значительно увеличить объемы собираемых данных, и за короткое время можно <<просканировать>> все обратное пространство кристалла.
Но в то же время качество снимаемых рефлексов при этом неумолимо страдает.
После съемки полученные пики как-то обрабатываются программами, работающими как <<черный ящик>>, и они же выдают значения ПЭЯ.
В процессе, могут уточняться инструментальные параметры прибора, и весь набор координат пиков по-сути аппроксимируется с помощью метода наименьших квадратов (МНК).
Такой подход и дает низкую точность определения ПЭЯ при монокристальной дифракции и использовании двумерных детекторов.

Для точного определения межатомных расстояний после расшифровки структуры кристалла, важны не только относительные координаты атомов в ячейке $(x, y, z)$, но и ПЭЯ, причем с относительной точностью не хуже чем относительные координаты атомов.
Поэтому метод, который бы позволил без больших затрат точно определять ПЭЯ на первом этапе рентгеноструктурного анализа (РСтА) с тем же образцом и на той же установке позволил бы улучшить его результаты.

Также точный метод определения ПЭЯ можно использовать и для калибровки других дифракционных установок.
Хорошо измерив ПЭЯ на одном приборе, можно использовать тот же кристалл на другом приборе для его калибровки.
Впоследствии это позволит более объективно сравнивать результаты, получаемые на двух разных установках.

Все написанное выше является является мотивацией к теме работы --- разработке прецизионного метода определения ПЭЯ для малых монокристаллов на установках, оснащенных матричным детектором.
Малость монокристалла возникает из-за требования схожести образца с тем, что используется в РСтА.