\section{Теоретическая часть}

Главной особенностью современных дифракционных установок является, как уже было отмечено во введении, использование двумерных детекторов вместо точечных.
Поэтому классическое измерение угла $\omega$, при котором наблюдается максимум интенсивности дифрагированного луча оказывается крайне неэффективным.
Ведь при этом пришлось бы делать большое множество изображений с малым шагом, вычислять интегральную интенсивность луча, суммируя значения в определенной области детектора, а затем находить математическими методами угол, при котором и наблюдается максимум.
Такая схема проведения эксперимента является очень неэффективной и, в принципе, неестественной для современных дифрактометров, ведь сейчас обычно производится съемка при равномерно вращающемся кристалле в небольшом диапазоне углов, или, так называемое, сканирование.
Предлагается использовать именно результат такого сканирования для точного определения угла дифракции.

\subsection{Описание сканирования}

Рассмотрим идеализированную модель эксперимента, где пучок идеально коллимирован и монохроматичен, кристалл совершенен, а детектор позволяет абсолютно точно измерить интенсивность падающего на него излучения в каждой точке.
В таком случае, в кинематической теории дифракции, отражение от выбранной плоскости кристалла будет наблюдаться может только при дискретном наборе углов сканирования $\omega$.
Дифрагированный пучок же будет иметь нулевую расходимость, как и первичный.
Понять это можно, например, рассматривая уравнение Вульфа-Брэгга в более общем виде, чем~(\ref{eq:bragg}):
\begin{equation} \label{eq:bragg_general}
    \vec{k} + \vec{q} = k \vec{n}
\end{equation}
где $\vec{k}$ --- волновой вектор первичного пучка, $\vec{q}$ --- вектор рассеяния, равный по величине вектору обратной решетки выбранной плоскости, который вращается в процессе сканирования из-за вращения самого кристалла, $k$ --- длина вектора $\vec{k}$, а $\vec{n}$ --- единичный вектор направления дифрагированного луча.
В используемой модели вектор $\vec{k}$ является постоянным, а вектор $\vec{n}$ произвольным, так как двумерный детектор может регистрировать двумерное множество направлений $\vec{n}$.
Поэтому, можно возвести обе части векторного уравнения~(\ref{eq:bragg_general}) в квадрат и получить скаляры.
\begin{equation} \label{eq:squared_bragg}
    2(\vec{q}(\omega) \cdot \vec{k}) + q^2 = 0
\end{equation}
Теперь, так как сканирование производится вдоль одной оси, то вектор $\vec{q}$ зависит только от одной переменной --- угла сканирования, а значит решениями получившегося уравнения окажется дискретный набор углов.
В итоге картина интенсивности на детекторе при сканировании в области, захватывающей один угол $\omega$, при котором наблюдается отражение будет представлять собой просто дельта-функцию.

В реальности же, конечно, дельта-функция уширяется и представляет собой локализованный двумерный профиль в виде пика, или нескольких пиков (например, пары, при разделении дублета $K\alpha_{1,2}$).
Форма этого профиля определяется множеством различных факторов: спектром источника, структурой и формой кристалла, параметрами детектора, сканирования, и так далее.
Все это влияние принято учитывать в виде так называемой инструментальной функции.
Информация о ней позволяет из экспериментально полученных данных точно определить положение дифракционного профиля на детекторе, соответствующее идеализированной модели.
Но теоретическое вычисление инструментальной функции чаще всего или невозможно или очень трудоемко.
Поэтому можно либо измерять ее экспериментально, либо аппроксимировать эмпирически полученными функциями.

\subsection{Геометрия установки}

Чтобы более строго описать процесс измерения необходимо ввести набор понятий, которые определяют геометрию установки, и используются в работе в дальнейшем.

Уже была введена ось $\omega$, при вращении кристалла вокруг которой производится сканирование.
Ей же соответствует и одноименный угол.
Также в требованиях к установке было отмечено, что детектор должен быть установлен на гониометре, допускающем вращение вокруг образца.
Соответствующая плоскость вращения и одноименный с ней угол будут называться $2\theta_D$.
Ось вращения детектора названа так из-за того, что она юстируется специально, чтобы первичный пучок находился точно параллельно ей.
Угол $2\theta_D$ также для однозначности будем нормировать на интервал $(-180\degree, 180\degree)$.

Также необходимо ввести систему координат детектора.
Обычно она появляется естественным образом, так как детектор образует матрицу, и координатами в этом случае являются целочисленные индексы пикселей на ней.
Сейчас же введем координаты детектора $X$ и $Y$ таким образом, чтобы ось $X$ была параллельна плоскости $2\theta_D$, а ось $Y$ --- перпендикулярна ей.

\subsection{Схема Бонда}

Теперь, по аналогии со съемкой одного рефлекса в двух симметричных положениях в методе Бонда, проведем два $\omega$-сканирования для таких же симметричных положений.
Но в нашем случае, их логично расположить в плоскости $2\theta_D$, так как при полном  