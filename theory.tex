\section{Теоретическая часть}

Главной особенностью современных дифракционных установок является, как уже было отмечено во введении, использование двумерных детекторов заместо точечных.
Поэтому классическое измерение угла $\omega$, при котором наблюдается максимум интенсивности дифрагированного луча оказывается крайне неэффективным.
Ведь при этом пришлось бы делать большое множество изображений с малым шагом, вычислять интегральную интенсивность луча, суммируя значения в определенной области детектора, а затем находить математическими методами угол, при котором и наблюдается максимум.
Такая схема проведения эксперимента является очень неестественной для современных дифрактометров в принципе, ведь сейчас обычно производится съемка при равномерно вращающемся кристалле в небольшом диапазоне углов, или, так называемое, сканирование.
Предлагается использовать именно результат такого сканирования для точного определения угла дифракции.