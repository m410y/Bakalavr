\section{Теоретическая часть}

Главной особенностью современных дифракционных установок является, как уже было отмечено во введении, использование двумерных детекторов вместо точечных.
Поэтому классическое измерение угла $\omega$, при котором наблюдается максимум интенсивности дифрагированного луча оказывается крайне неэффективным.
Ведь при этом пришлось бы делать большое множество изображений с малым шагом, вычислять интегральную интенсивность луча, суммируя значения в определенной области детектора, а затем находить математическими методами угол, при котором и наблюдается максимум.
Такая схема проведения эксперимента является очень неестественной для современных дифрактометров в принципе, ведь сейчас обычно производится съемка при равномерно вращающемся кристалле в небольшом диапазоне углов, или, так называемое, сканирование.
Предлагается использовать именно результат такого сканирования для точного определения угла дифракции.

\subsection{Модель сканирования}

Далее использоваться будет только кинематическая теория дифракции.
Рассмотрим идеализированную модель эксперимента, где пучок идеально коллимирован и монохроматичен, кристалл совершенен, а детектор позволяет абсолютно точно измерить интенсивность падающего на него излучения в каждой точке.
В таком случае, отражение от выбранной плоскости кристалла будет наблюдаться может только при дискретном наборе углов сканирования $\omega$.
Дифрагированный пучок же будет иметь нулевую расходимость, как и первичный.
Понять это можно, например, рассматривая уравнение Вульфа-Брэгга в более общем виде, чем~(\ref{eq:bragg}):
\begin{equation} \label{eq:bragg_general}
    \vec{k} + \vec{q} = k \vec{n}
\end{equation}
где $\vec{k}$ --- волновой вектор первичного пучка, $\vec{q}$ --- вектор рассеяния, равный по величине вектору обратной решетки выбранной плоскости, который вращается в процессе сканирования из-за вращения самого кристалла, $k$ --- длина вектора $\vec{k}$, а $\vec{n}$ --- единичный вектор направления дифрагированного луча.
В используемой модели вектор $\vec{k}$ является постоянным, а вектор $\vec{n}$ произвольным, так как двумерный детектор может регистрировать двумерное множество направлений $\vec{n}$.
Поэтому, можно возвести обе части векторного уравнения~(\ref{eq:bragg_general}) в квадрат и получить скаляры.
\begin{equation} \label{eq:squared_bragg}
    2(\vec{q}(\omega) \cdot \vec{k}) + q^2 = 0
\end{equation}
Теперь, так как сканирование производится вдоль одной оси, то вектор $\vec{q}$ зависит только от одной переменной --- угла сканирования, а значит решениями получившегося уравнения окажется дискретный набор углов.
В итоге картина интенсивности на детекторе при сканировании в области, захватывающей один угол $\omega$, при котором наблюдается отражение будет представлять собой просто дельта-функцию.

В реальности же, конечно, дельта-функция уширяется и представляет собой локализованный двумерный пик, или несколько пиков (например, пару, при разделении дублета $K\alpha_{1,2}$).
Форма этих пиков определяется множеством различных факторов: спектром источника, структурой и формой кристалла, параметрами детектора, сканирования, и так далее.
Все это влияние принято учитывать в виде так называемой инструментальной функции.
Информация о ней позволяет из экспериментально полученных данных точно определить положение дифракционного пика не детекторе, соответствующее идеализированной модели.
Теоретическое вычисление инструментальной функции чаще всего или невозможно или очень трудоемко.
Поэтому можно либо измерять ее экспериментально, либо аппроксимировать эмпирически полученными функциями.

\subsection{Схема Бонда}

Рассмотрим как выглядел бы аналог метода Бонда при использовании двумерного детектора.
Изначально она представляет собой съемку дифракции от кристалла в двух симметричных положениях.
Угол детектора будем обозначать как $2\theta_D$ и нормируем его на интервал $(-\pi, \pi)$.
Аналогично базовому методу тогда можно провести два $\omega$-сканирования в каждом из положений детектора $2\theta_D = \pm 2\theta$, где $2\theta$ --- это брэгговский угол, фигурирующий в выражении~(\ref{eq:bragg}).
На каждом из двух полученных изображений будет двумерный пик, как уже было описано выше.
В идеальном случае, эти пики будут располагаться в одной и той же точке детектора и будут зеркальным отражением друг друга.
Но при небольшом отклонении $2\theta_D$ от идеального $2\theta$, они окажутся смещенными.
Определя это смещение, можно вычислить поправку к углу $2\theta_D$, позволяющую получить точное значение $2\theta$.
Ее можно получить достаточно просто.


Раскладывая до первого порядка зависимость координаты пика 
