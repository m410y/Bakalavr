\section{Выводы}

Выполненное исследование показало ряд недостатков при реализации схемы Бонда на современном монокристальном дифрактометре, в первую очередь --- это ограничение углов $2\theta_D$ и большая полуширина рефлексов.
Большие значения FWHM $0.25-0.30\degree$ (см. табл. 3 Приложения) обусловлены расходимостью первичного пучка.
Несмотря на это, при наличии кристалла, пригодного для проведения РСтА, без особых затруднений можно проводить измерения $d$ с вполне приемлемой относительной ошибкой $5\cdot10^{-5}$.
Частично понизить это значение можно путем установки дополнительного детектора, который может иметь гораздо меньшие размеры, т.к. его задача --- зафиксировать единичное отражение.
Кроме этого, желательно, чтобы дополнительный детектор имел меньшие размеры пикселя, т.к. большее число точек на профиле отражения должно привести к повышению точности измерения угла $2\theta$.

Также в настоящей работе:
\begin{itemize}
    \item разработана методика уточнения ПЭЯ малых монокристаллов на основе схемы Бонда на серийном лабораторном дифрактометре;
    \item написана программа, позволяющая рассчитывать углы $\varphi$ и $\omega$ для выведения рефлексов в отражающие положения и обрабатывать результаты;
    \item методика протестирована на эталонных монокристаллах Si и Ge. Показано, что относительная погрешность измерений ПЭЯ не хуже $10^{-5}$;
    \item в схеме Бонда изучены 5 монокристаллов состава \YEu{}. Определены ПЭЯ и оценен интервал значений $x$ от $0.34$ до $0.40$. Сделан вывод о неоднородности синтезированного продукта.
\end{itemize}
